\chapter{Grundlagen}%

\label{cha:Schluss}

In dieser Arbeit werden einige Technologien genutzt um die Problemstellung zu lösen.
Die wichtigsten dieser Technologien werden hier erklärt um das weitere Verständniss des Lesers zu gewährleisten. 


\section{Vier Gewinnt}
Vier Gewinnt ist ein Spiel für 2 Spieler in dem abwechselnd Spielsteine in ein vertikales Spielfeld eingeworfen werden. Jeder Spieler kann sich in seinem Zug zwischen einem von sieben Einwurflöchern entscheiden. Wird ein Spielstein eingeworfen so fällt dieser auf die niedrigste der sechs Positionen die noch nicht durch andere Spielsteine gefüllt ist.Hierdurch ergibt sich ein Spielfeld mit 42 Feldern.  Sind schon sechs Spielsteine in ein bestimmtest Einwurfloch gesteckt wurden so darf hier kein weiterer Spielstein eingeworfen werden. Ein Spieler gewinnt das Spiel wenn er es geschafft  hat, dass sich vier seiner Spielsteine in einer Reihe befinden. Eine Reihe kann horizontal, vertikal oder diagonal gebildet werden. Werden alle 42 Positionen mit Spielsteinen befüllt ohne eine Reihe von vier Steinen des selben Spielers zu bilden geht die Partie unentschieden aus.
%TODO BILD


\section{Minmax Algorithmus}
Ein Minmax Algorithmus beschreibt einen rekursiven Ansatz zum finden einer optimalen Lösung für Probleme von zwei Parteien mit widersetzlich Zielen. Dies sind in der Regel Nullsummenspiele mit perfekter Information, es gibt also für jeden Gewinn des einen Spielers genauso viel Verlust für den anderen und es gibt kein Spielelement, dass nur für einen Spieler einsehbar ist. Hierbei werden in jedem Rekursionsschritt alle möglichen Züge des derzeitigen Akteurs betrachtet und bewertet. Wenn der derzeitige Zug nicht zum Gewinn führt, findet eine Bewertung durch das Betrachten des nächsten Rekursionsschrittes statt. Der Name Minmax ergibt sich durch das Betrachten der abwechselnden Züge von den Akteuren wobei jeweils der maximal beste Zug für den einen der minimal beste Zug für den anderen bedeutet.  Dieses verfahren führt bei einfachen Spielen wie Tic-Tac-Toe zu einem \colorbox{red!30}{TODO}.\\

\subsection{Negamax}
Negamax ist eine Variante des Minmax Algorithmus die \colorbox{red!30}{TODO}.

\section{Reinforcement Learning}
Reinforcement Learning (Bestärkendes Lernen) ist eine Maschine-Learning Methode bei der ein Agent mit Hilfe eines Environments, Policy, State, Actions und eines Rewards lernt eine ihm gegebene Aufgabe zu lösen.
Die Aufgabe ist in diesem Fall das Spiel Vier Gewinnt zu meistern.

\subsection{Environment}
Das Environment beschreibt das gesamte Problem dass der Agent versucht zu lösen.


\subsection{Agent}
Bei dem Agenten handelt es sich um den Acteur der die ihm gegebene Aufgabe meistern soll. Er sieht das Environment und wählt mittels der Policy die ihm am besten erscheinende Action aus.


\subsection{Policy}
Mit der Policy wird das Verhalten beschrieben nach dem der Agent entscheidet welche der möglichen Actions die aktuell beste ist um den Reward zu maximieren.\\
\colorbox{red!30}{TODO} %welche gibt es so?

\subsection{State}
Der State beschreibt den aktuellen Zustand des Environments.

\subsection{Actions}
Actions sind die möglichen Aktionen zwischen denen sich der Agent je nach State entscheiden muss.

\subsection{Reward}

\colorbox{red!30}{TODO} %ausführen

\subsection{Q-Funktion}
\colorbox{red!30}{TODO} %ausführen

\subsection{psychologischer Hintergrund}
Reinforcement Learning funktioniert nach dem Prinzip von Trial-and-Error.\\
\colorbox{red!30}{TODO} %ausführen



\section{Software}
Um die Problemstellung zu bearbeiten wurde eine Implementierung des Spiels in Python genutzt welches als ein Gym Environment fungiert. An diesem Environment trainiert dann ein Agenten der mit Hilfe von  Keras erschaffen wurde.
\colorbox{red!30}{TODO genaue versionen}

\subsection{Python}
Als Programmiersprache wurde sich für Python entschieden da ein gutes Vorwissen in dieser Sprache vorhanden war und alle für diese Arbeit wichtigen Packages in dieser Programmiersprache existieren.
Der Schöpfer von Python beschreibt es selbst als einfach zu erlernende objektorientierte Programmiersprache welche Leistung mit einer klaren Syntax verbindet.\\
\colorbox{red!30}{TODO https://dl.acm.org/doi/book/10.5555/1593511}

\subsection{Tensorflow}
Bei Tensorflow handelt es sich um ein open source Framework für Maschine-Learning welches ursprünglich für den internen Bedarf bei Google, für zum Beispiel Spracherkennung oder Google Maps, entwickelt wurde. 
\colorbox{red!30}{TODO mehr?}

\subsection{Keras}
Keras ist eine open source Bibliothek die auf Tensorflow aufbaut welche die möglichkeit bietet vereinfacht neuronale Netze zu erstellen um diese in reinforcement Learning Algorithmen zu benutzen.
\colorbox{red!30}{TODO mehr?}

\subsection{OpenAI Gym}
Gym ist ein Werkzeug zum entwickeln und vergleichen von reinforcement learning Algorithmen welches von OpenAI, einem Forschungslabor aus San Francisco, zur Verfügung gestellt wird. Es bietet eine  \colorbox{red!30}{angleichende  Schnittstelle} zwischen dem Environment und dem Agenten. Hierdurch können neue Agenten und Environments nach gewissen Vorgaben erstellt werden. Somit ist es mit Gym einfacher möglich verschiedene Agenten an einem Environment zu Trainieren und diese zu vergleichen oder den selben Agenten an verschiedenen Environments auf seine Anpassungsfähigkeit zu testen.

