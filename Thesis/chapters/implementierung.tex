\chapter{Implementierung}%

\label{cha:Implementierung}

Vorgehensweise bei der Implementierung
%was man alles anpassen muss

\colorbox{red!30}{DQN Algoritmus zum lernen des Spiels}\\
\colorbox{red!30}{Trainiert gegen Random/Minmax}\\
\colorbox{red!30}{Auswertung: ???}\\


\section{Environment}
\colorbox{red!30}{TODO}In Vier Gewinnt ist dies das Spielfeld und der Gegner. 

\section{Agent}
\colorbox{red!30}{TODO} Hier ist der Agent einer der beiden Spieler von Vier Gewinnt.

\section{Policy}
\colorbox{red!30}{TODO}

\section{State}
\colorbox{red!30}{TODO} Für Vier Gewinnt beschreibt der State welche der 42 Felder mit welchen Steinen befüllt sind. 

\section{Actions}
\colorbox{red!30}{TODO} Die Actions in Vier Gewinnt beschreiben die sieben Löcher in die der Agent Steine werfen kann.

\section{Reward}
\colorbox{red!30}{TODO} Der Reward wird immer zum Ende einer Partie ausgegeben und beträgt beim Gewinn 1 beim Verlieren -1 und wenn die Partie unentschieden ausgeht 0. Neuer Ansatz:
nach 4 Zügen gewonnen -> 1 und nach 21 zügen unentschieden -> 0  allles andere dazwischen und bei verlust *-1. also reward = 4/anzahl züge *winner?

\subsection{Q-Funktion}
\colorbox{red!30}{TODO} %ausführen