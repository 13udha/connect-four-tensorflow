\chapter{Einleitung}
\label{cha:Einleitung}

%Was will ich machen?
%In der Bachelorarbeit wurde ein Programm entwickelt, das mit Hilfe von reinforcement learning trainiert um das Spiel Vier Gewinnt zu erlernen. 
%Es wird ein reinforcement learning Ansatz gewählt, da hierdurch automatisiert individuelle Gegner geschaffen werden können, die das Können unterschiedlicher Spieler widerspiegeln.%???
%Spiele müssen competetiv bleiben um spannend zu sein daher anpassen wichtig
%Klassische spiele digital umsetzen bsp alpha go


\section{Motivation}
%Warum will ich das machen?
%Warum spielen wir? Wir spielen um Fähigkeiten zu erlernen, ein Kind lernt Objekte nach Formen zu sortieren oder eine Katze lernt spielerisch das Jagen von Beute. 

Spiele machen Spaß und können genutzt werden um neue Fähigkeiten zu lernen oder verbessern. Das Lernen beim Spielen funktioniert dann am besten, wenn das Spiel einen gewissen Schwierigkeitsgrad hat, welcher aber noch zu bewältigen sein muss. Ist das Spiel zu einfach, wird der Spieler gelangweilt, ist es zu schwer, wird er frustriert. In beiden Fällen sucht er sich wahrscheinlich sehr bald eine andere Beschäftigung. Hiergegen kann es helfen einen anderen Spieler zu haben, welcher ein ähnliches Level an Erfahrung mitbringt. Wenn nun die Rolle des Gegners nicht von einem Menschen besetzt werden kann ist es naheliegend einen Computergegner zu erschaffen. Solch ein Computergegner wird normalerweise mit Algorithmen umgesetzt, die einen optimalen Zug errechnen. Seit einigen Jahren ist es, durch den Vortschritt in Hardware und Wissenschaft, möglich Computergegner zu erschaffen, die ihr können durch Neuronale Netze antrainiert haben. 

% Da aber jeder Mensch, egal wie erfahren, das Spiel spielen soll ist es oft hilfreich einen Gegner zu haben der ein ähnliches Level an Erfahrung mitbringt.
%Um den bestmöglichen Effekt zu haben muss das Spiel einen gewissen Schwierigkeitsgrad haben der aber noch zu bewältigen sein muss.

% Die Frage die ich in dieser Bachelorarbeit versuche zu beantworten ist:   Ist es möglich, das Spiel Vier Gewinnt durch reinforcement learning zu erlernen und wie hilfreich ist dabei das Lernen vom Gegner.
%Gehören schon immer zum menschlichen miteinander... barrieren abbauen


\section{Problemstellung}
%Was steht zur Verfügung? Was ist das Problem genau(mathematisch)? Random vs best move every turn~
%genau das spiel beschreiben und was das problem dann ist und wie das gelöst werden kann

Computergegner die mit Hilfe von Neuronalen Netzen trrainiert wurden, haben das Problem, dass sie nur gut werden können wenn es auch einen guten Gegner gibt oder eine enorme Menge an Spielaufzeichnungen existiert, an denen sie trainieren können. Da dieses training erst richtigen Effekt zeigt, wenn sehr viele Spiele gespielt wurden, ist es nicht wirklich möglich einen Menschen als Trainer zu nutzen. Also wird hier normalerweise auf einen weiteren Computergegner zurückgegriffen, welcher das Spiel beherrscht ohne dies mit Neuronalen Netzen erlernt zu haben. Es wäre schön wenn die Anzahl der benötigten Spiele beim training reduziert werden könnte, um schneller bessere Ergebnisse zu erziehlen oder sogar nur mit einem Mensch lernen zu können.



%Ein Computergegner der für das Spiel Vier Gewinnt erschaffen wird kann auf verschiedenen Algorithmen basieren er könnte zum Beispiel in jedem Zug einfach zufällig eines der sieben möglichen Einwurflöcher bedienen oder er könnte durch einen Minimax Algorithmus zu jedem Zeitpunkt den bestmöglichen Zug errechnen. 

\section{Umsetzung}

Meine Idee ist es, das Training durch das Mitlernen der Züge des Gegners zu verbessern. Dies erscheint sinnvoll, da der Gegner das Spiel schon beherrscht und daher tendenziell gute Züge machen wird. 

%Ein Lösungsansatz für diese Problem sind selbstlernende Algorithmen die mit Hilfe von reinforement learning und neuronalen Netzen selbstständig ähnlich wie ein Mensch erlernen wie das Spiel zu gewinnen ist. Hierraus können dann automatisch Computergegner generiert werden. 
%Wie gehe ich vor?
%Minimax algorithmus als gegner erstellen und an dem dann trainieren
%minimax da für dieses spiel möglich













