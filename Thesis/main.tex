%%% File encoding: UTF-8
%%% äöüÄÖÜß  <-- keine deutschen Umlaute hier? UTF-faehigen Editor verwenden!

%%% Magic Comments zum Setzen der korrekten Parameter in kompatiblen IDEs
% !TeX encoding = utf8
% !TeX program = pdflatex 
% !TeX spellcheck = de_DE
% !BIB program = biber

\documentclass[bachelor,german]{hgbthesis}
% Zulässige Optionen in [..]: 
%   Typ der Arbeit: diploma, master (default), bachelor, internship 
%   Hauptsprache: german (default), english
%%%----------------------------------------------------------

\RequirePackage[utf8]{inputenc}		% bei der Verw. von lualatex oder xelatex entfernen!
\usepackage{amsmath}
\usepackage{subfig}
\graphicspath{{images/}}    % Verzeichnis mit Bildern und Grafiken
\logofile{logo.png}				% Logo-Datei = images/logo.pdf (\logofile{}, wenn kein Logo gewünscht)
\bibliography{references}  	% Biblatex-Literaturdatei (references.bib)

%%%----------------------------------------------------------
% Angaben für die Titelei (Titelseite, Erklärung etc.)
%%%----------------------------------------------------------

%%% Einträge für ALLE Arbeiten: -----------------------------
\title{Computer spielen Computerspiele mit Hilfe von reinforcement learning am Beispiel Vier Gewinnt}
\author{Pablo Lubitz}
\programname{Informatik}
\placeofstudy{Bremen}
\dateofsubmission{2020}{03}{09}	% {YYYY}{MM}{DD}

%%% Zusätzlich für eine Bachelorarbeit: ---------------------
%\thesisnumber{2925704-A}   % Stud-ID, z.B. 1310238045-A  
% (A = 1. Bachelorarbeit)
\semester{Sommersemester 2019} 
\coursetitle{Bachelor Arbeit} 
\advisor{Holger Schultheis, Thomas Barkowsky}

%%% Restriktive Lizenformel anstatt CC (nur für Typ master) -
%\strictlicense

%%%----------------------------------------------------------
\begin{document}
%%%----------------------------------------------------------

%%%----------------------------------------------------------
\frontmatter                    % Titelei (röm. Seitenzahlen)
%%%----------------------------------------------------------

\maketitle
\tableofcontents

\chapter{Vorwort} 	% engl. Preface


TODO




 % Optional. Ggf. weglassen
\chapter{Kurzfassung}

In dieser Bachelorarbeit wurde sich mit dem Thema Reinforcement-Learning in Spielen beschäftigt. Es wurde eine Implementierung dieses Verfahrens für das Spiel Vier Gewinnt genutzt. Mit der  Implementation wurde getestet, ob dieses Verfahren durch das Beobachten des gegnerischen Verhaltens optimiert werden kann. Hierbei wurde eine minimale Verbesserung festgestellt.
		
\chapter{Abstract}
In this bachelor thesis we are exploring reinforcement learning in the area of games. Therfore an implementation of this procedure was used for the game Connect Four. Furthermore a test was made to proof that the implementation could be optimized with the addition of a way to learn what your enemie did. This was shown to have a minimal positive impact.

			

%%%----------------------------------------------------------
\mainmatter          % Hauptteil (ab hier arab. Seitenzahlen)
%%%----------------------------------------------------------

\chapter{Einleitung}
\label{cha:Einleitung}

%Was will ich machen?
%In der Bachelorarbeit wurde ein Programm entwickelt, das mit Hilfe von reinforcement learning trainiert um das Spiel Vier Gewinnt zu erlernen. 
%Es wird ein reinforcement learning Ansatz gewählt, da hierdurch automatisiert individuelle Gegner geschaffen werden können, die das Können unterschiedlicher Spieler widerspiegeln.%???
%Spiele müssen competetiv bleiben um spannend zu sein daher anpassen wichtig
%Klassische spiele digital umsetzen bsp alpha go


\section{Motivation}
%Warum will ich das machen?
%Warum spielen wir? Wir spielen um Fähigkeiten zu erlernen, ein Kind lernt Objekte nach Formen zu sortieren oder eine Katze lernt spielerisch das Jagen von Beute. 

Spiele machen Spaß und können genutzt werden um neue Fähigkeiten zu lernen oder verbessern. Das Lernen beim Spielen funktioniert dann am besten, wenn das Spiel einen gewissen Schwierigkeitsgrad hat, welcher aber noch zu bewältigen sein muss. Ist das Spiel zu einfach, wird der Spieler gelangweilt, ist es zu schwer, wird er frustriert. In beiden Fällen sucht er sich wahrscheinlich sehr bald eine andere Beschäftigung. Hiergegen kann es helfen einen anderen Spieler zu haben, welcher ein ähnliches Level an Erfahrung mitbringt. Wenn nun die Rolle des Gegners nicht von einem Menschen besetzt werden kann ist es naheliegend einen Computergegner zu erschaffen. Solch ein Computergegner wird normalerweise mit Algorithmen umgesetzt, die einen optimalen Zug errechnen. Seit einigen Jahren ist es, durch den Vortschritt in Hardware und Wissenschaft, möglich Computergegner zu erschaffen, die ihr können durch Neuronale Netze antrainiert haben. 

% Da aber jeder Mensch, egal wie erfahren, das Spiel spielen soll ist es oft hilfreich einen Gegner zu haben der ein ähnliches Level an Erfahrung mitbringt.
%Um den bestmöglichen Effekt zu haben muss das Spiel einen gewissen Schwierigkeitsgrad haben der aber noch zu bewältigen sein muss.

% Die Frage die ich in dieser Bachelorarbeit versuche zu beantworten ist:   Ist es möglich, das Spiel Vier Gewinnt durch reinforcement learning zu erlernen und wie hilfreich ist dabei das Lernen vom Gegner.
%Gehören schon immer zum menschlichen miteinander... barrieren abbauen


\section{Problemstellung}
%Was steht zur Verfügung? Was ist das Problem genau(mathematisch)? Random vs best move every turn~
%genau das spiel beschreiben und was das problem dann ist und wie das gelöst werden kann

Computergegner die mit Hilfe von Neuronalen Netzen trrainiert wurden, haben das Problem, dass sie nur gut werden können wenn es auch einen guten Gegner gibt oder eine enorme Menge an Spielaufzeichnungen existiert, an denen sie trainieren können. Da dieses training erst richtigen Effekt zeigt, wenn sehr viele Spiele gespielt wurden, ist es nicht wirklich möglich einen Menschen als Trainer zu nutzen. Also wird hier normalerweise auf einen weiteren Computergegner zurückgegriffen, welcher das Spiel beherrscht ohne dies mit Neuronalen Netzen erlernt zu haben. Es wäre schön wenn die Anzahl der benötigten Spiele beim training reduziert werden könnte, um schneller bessere Ergebnisse zu erziehlen oder sogar nur mit einem Mensch lernen zu können.



%Ein Computergegner der für das Spiel Vier Gewinnt erschaffen wird kann auf verschiedenen Algorithmen basieren er könnte zum Beispiel in jedem Zug einfach zufällig eines der sieben möglichen Einwurflöcher bedienen oder er könnte durch einen Minimax Algorithmus zu jedem Zeitpunkt den bestmöglichen Zug errechnen. 

\section{Umsetzung}

Meine Idee ist es, das Training durch das Mitlernen der Züge des Gegners zu verbessern. Dies erscheint sinnvoll, da der Gegner das Spiel schon beherrscht und daher tendenziell gute Züge machen wird. 

%Ein Lösungsansatz für diese Problem sind selbstlernende Algorithmen die mit Hilfe von reinforement learning und neuronalen Netzen selbstständig ähnlich wie ein Mensch erlernen wie das Spiel zu gewinnen ist. Hierraus können dann automatisch Computergegner generiert werden. 
%Wie gehe ich vor?
%Minimax algorithmus als gegner erstellen und an dem dann trainieren
%minimax da für dieses spiel möglich














\chapter{Grundlagen}%

\label{cha:Schluss}

Eine Liste der genutzten Technologien.

\section{Reinforcement Learning}
\subsection{psychologischer Hintergrund}

\subsection{Agent}

\subsection{Environment}

\subsection{Policy}

\subsection{Reward}

\subsection{State}

\subsection{Actions}

\subsection{Q-Funktion}


\section{Vier Gewinnt}
Alles was es über das Spiel zu wissen gibt

\section{Python}
\section{Tensorflow}
\section{andere Software?}
\chapter{Konzept}%

\label{cha:Konzept}

Im Folgenden wird nun beschrieben, was das inhaltliche Ziel dieser Bachelorarbeit ist und wie vorgegangen werden soll um dieses Ziel zu erreichen.

\section{Zielsetzung}
Es soll am Beispiel des Spiels Vier Gewinnt gezeigt werden wie ein Agent funktioniert, welcher die Spielstrategie mit Hilfe von Reinforcement-Learning erlernt hat und ob diese Technik durch das Imitieren des Gegners verbessert werden kann.

\section{Methode}
%
Es wird eine digitale Version des Spiels Vier Gewinnt genutzt und eine Schnittstelle erschaffen die es ermöglicht auszuwählen, ob ein Mensch oder der Computer die Rolle der Spieler übernimmt. Dies ist wichtig, da ein selbstlernender Algorithmus viele Spiele durchlaufen muss, um einen Lernfortschritt aufzuzeigen. Somit kann in der Simulation trainiert werden, ohne jedes einzelne Spiel gegen einen Menschen spielen zu müssen. Spielt der Computer, soll zwischen verschiedenen Strategien ausgewählt werden können, welche unterschiedlich effektiv sind. Die Strategien sind: komplett zufällige Züge wählen, immer so weit wie möglich links ins Spielfeld werfen und ein Minimax-Algorithmus, welcher verschieden starke Varianten unterstützt. Es wurde sich für diese Strategien entschieden, da bei den einfachen Strategien das Lernverhalten besser zu erkennen ist. Für den Minimax-Algorithmus wurde sich entschieden um dann das Lernverhalten gegen einen sehr guten Gegner zu zeigen.
An diesen Strategien soll der Agent mit verschiedenen Voraussetzungen trainieren, um dann vergleichen zu können, was besser und schlechter beim Lernen hilft.
%

%Es wird ein Programm geschrieben, das durch reinforcement learning erlernt das Spiel Vier Gewinnt zu meistern. Um dies zu ermöglichen wird ein Gegner genutzt, der möglichst optimal spielt. Hierfür wird ein Minimax Algorithmus sorgen. Es wird eine optimierung der Reward-Funktion stattfinden um das Lernverhalten zu verbessern. Das Verhalten des guten Gegners soll mit in den Lernprozess des Agenten einfließen. Um den Lernfortschritt besser beurteilen zu können wird der Minimax-Algorithmus um eine Wahrscheinlichkeitsverteilung erweitert. Hierdurch soll dann das erlernte Verhalten an leicht schlechteren Gegnern getestet werden können.
% old
\newpage

\section{Evaluierung}
Je nachdem wie gut der selbstlernende Agent nach dem Trainieren gegen die unterschiedlich effektiven Strategien ist, zeigt sich dann, wie gut das Reinforcement-Learning mit den verschiedenen Voraussetzungen funktioniert hat. Somit kann dann gezeigt werden welche Veränderungen besser sind, um einen guten Reinforcement-Learning Agenten zu erschaffen. Es sollte sich dann auch zeigen ob das Lernen des Verhalten des Gegners zu einer Verbesserung des Lernverhaltens des Reinforcement-Learning-Agenten führt.


\chapter{Implementierung}%

\label{cha:Implementierung}

Vorgehensweise bei der Implementierung
%was man alles anpassen muss

\colorbox{red!30}{DQN Algoritmus zum lernen des Spiels}\\
\colorbox{red!30}{Trainiert gegen Random/Minmax}\\
\colorbox{red!30}{Auswertung: ???}\\


\section{Environment}
\colorbox{red!30}{TODO}In Vier Gewinnt ist dies das Spielfeld und der Gegner. 

\section{Agent}
\colorbox{red!30}{TODO} Hier ist der Agent einer der beiden Spieler von Vier Gewinnt.

\section{Policy}
\colorbox{red!30}{TODO}

\section{State}
\colorbox{red!30}{TODO} Für Vier Gewinnt beschreibt der State welche der 42 Felder mit welchen Steinen befüllt sind. 

\section{Actions}
\colorbox{red!30}{TODO} Die Actions in Vier Gewinnt beschreiben die sieben Löcher in die der Agent Steine werfen kann.

\section{Reward}
\colorbox{red!30}{TODO} Der Reward wird immer zum Ende einer Partie ausgegeben und beträgt beim Gewinn 1 beim Verlieren -1 und wenn die Partie unentschieden ausgeht 0.

\subsection{Q-Funktion}
\colorbox{red!30}{TODO} %ausführen
\chapter{Evaluierung}%

\label{cha:Eval}

Um die Effektivität der verschiedenen Versionen des reinforcement learning Agenten zu Evaluieren werden diese hier verglichen. Weitergehend werden alle Versionen noch mit diesen Anzahlen von Episoden trainiert um die Effektivität über längere trainingsabläufe zu vergleichen.

\colorbox{red!30}{jede veränderung auflisten und verschieden lange lernperioden bennen}


\section{Auswertung}
\colorbox{red!30}{Ergebnisse in Text übersetzen-> Graph, Durchschnitt,...}

\colorbox{red!30}{Schlüsse die aus den Daten gezogen werden können -> Signifikantstest?}





\chapter{Fazit}%

\label{cha:Fazit}


Mit dieser Bachelorarbeit sollte geprüft werden, ob das Lernen der Züge des Gegners eine positive Auswirkung auf das Lernverhalten des Reinforcement-Learning-Agenten hat.
Hierzu wurde die für diese Arbeit entwickelte Implementierung des Spiels Vier Gewinnt genutzt. In der Evaluierung konnte beim Vergleich der verschiedenen Versionen (mit und ohne Lernen der gegnerischen Züge) an Gegnern mit einfachen Strategien ein positiver Effekt festgestellt werden. Des weiteren wurde gezeigt, dass die  gewählte Reward-Funktion angemessen war und somit zur Verbesserung des Lernverhaltens beigetragen hat.
Diese Verbesserungen konnten vorerst leider nur am Beispiel von Gegnern mit schlechten und sehr einfach zielführenden Strategien, jedoch nicht am Gegner mit sehr guter Strategie gezeigt werden.
Die Vermutung liegt nahe, dass dieses Verhalten auch bei der sehr guten Strategie eintritt, wenn eine längere Lernepoche absolviert wird.
Da das Training des Agenten, welches in dieser Arbeit gemacht wurde, trotz der vorgenommenen Optimierungen schon sehr viel Zeit in Anspruch genommen hat, wurde davon abgesehen das Verhalten in längeren Lernepochen zu beobachten. Es spricht aber tendeziell nichts dagegen, dass sich ein Lernerfolg gegen den Minimax-Algorithmus zeigen kann.\\

In weiteren Untersuchungen könnte versucht werden, ob sich das hier gezeigte positive Verhalten auch beim Trainieren mit besseren Gegnern zeigen lassen kann. 
Hierfür müsste dann gegebenenfalls die Problemstellung vereinfacht oder der Algorithmus verbessert werden, sodass längere Lernepochen in überschaubarer Zeit absolviert werden können.



%Negamax with alpha beta pruning and transposition tables TODO das erwähnen

%%%----------------------------------------------------------
\appendix                                            % Anhang 
%%%----------------------------------------------------------


%%%----------------------------------------------------------
\MakeBibliography                        % Quellenverzeichnis
%%%----------------------------------------------------------

%%% Messbox zur Druckkontrolle ------------------------------
\include{back/messbox}

%%%----------------------------------------------------------
\end{document}
%%%----------------------------------------------------------