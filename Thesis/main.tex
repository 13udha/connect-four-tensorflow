%%% File encoding: UTF-8
%%% äöüÄÖÜß  <-- keine deutschen Umlaute hier? UTF-faehigen Editor verwenden!

%%% Magic Comments zum Setzen der korrekten Parameter in kompatiblen IDEs
% !TeX encoding = utf8
% !TeX program = pdflatex 
% !TeX spellcheck = de_DE
% !BIB program = biber

\documentclass[bachelor,german]{hgbthesis}
% Zulässige Optionen in [..]: 
%   Typ der Arbeit: diploma, master (default), bachelor, internship 
%   Hauptsprache: german (default), english
%%%----------------------------------------------------------

\RequirePackage[utf8]{inputenc}		% bei der Verw. von lualatex oder xelatex entfernen!
\usepackage{amsmath}
\usepackage{subfig}
\graphicspath{{images/}}    % Verzeichnis mit Bildern und Grafiken
\logofile{logo.png}				% Logo-Datei = images/logo.pdf (\logofile{}, wenn kein Logo gewünscht)
\bibliography{references}  	% Biblatex-Literaturdatei (references.bib)

%%%----------------------------------------------------------
% Angaben für die Titelei (Titelseite, Erklärung etc.)
%%%----------------------------------------------------------

%%% Einträge für ALLE Arbeiten: -----------------------------
\title{Computer spielen Computerspiele mit Hilfe von reinforcement learning am Beispiel Vier Gewinnt}
\author{Pablo Lubitz}
\programname{Informatik}
\placeofstudy{Bremen}
\dateofsubmission{2020}{03}{09}	% {YYYY}{MM}{DD}

%%% Zusätzlich für eine Bachelorarbeit: ---------------------
%\thesisnumber{2925704-A}   % Stud-ID, z.B. 1310238045-A  
% (A = 1. Bachelorarbeit)
\semester{Sommersemester 2019} 
\coursetitle{Bachelor Arbeit} 
\advisor{Holger Schultheis, Thomas Barkowsky}

%%% Restriktive Lizenformel anstatt CC (nur für Typ master) -
%\strictlicense

%%%----------------------------------------------------------
\begin{document}
%%%----------------------------------------------------------

%%%----------------------------------------------------------
\frontmatter                    % Titelei (röm. Seitenzahlen)
%%%----------------------------------------------------------

\maketitle
\tableofcontents

%\chapter{Vorwort} 	% engl. Preface


TODO




 % Optional. Ggf. weglassen
\chapter{Kurzfassung}

TODO		
\chapter{Abstract}

TODO engl.
			

%%%----------------------------------------------------------
\mainmatter          % Hauptteil (ab hier arab. Seitenzahlen)
%%%----------------------------------------------------------

\chapter{Einleitung}
\label{cha:Einleitung}

%Was will ich machen?
In der Bachelorarbeit wurde ein Programm entwickelt, das mit Hilfe von reinforcement learning trainiert um das Spiel Vier Gewinnt zu erlernen. 
Es wird ein reinforcement learning Ansatz gewählt, da hierdurch automatisiert individuelle Gegner geschaffen werden können, die das Können unterschiedlicher Spieler widerspiegeln.%???
%Spiele müssen competetiv bleiben um spannend zu sein daher anpassen wichtig
%Klassische spiele digital umsetzen bsp alpha go


\section{Motivation}
%Warum will ich das machen?
Warum spielen wir? Wir spielen um Fähigkeiten zu erlernen, ein Kind lernt Objekte nach Formen zu sortieren oder eine Katze lernt spielerisch das Jagen von Beute. Doch auch erwachsene Menschen spielen noch und können hierdurch ihre Fähigkeiten verbessern.  Um den bestmöglichen Effekt zu haben muss das Spiel einen gewissen Schwierigkeitsgrad haben der aber noch zu bewältigen sein muss. Da aber jeder Mensch, egal wie erfahren, das Spiel spielen soll ist es oft Hilfreich einen Gegner zu haben der ein ähnliches Level an Erfahrung mitbringt. Wenn nun diese Rolle des Gegners nicht von einem Menschen besetzt werden kann ist es naheliegend einen Computergegner zu erschaffen. Die Frage die ich in dieser Bachelorarbeit versuche zu beantworten ist: Ist es möglich durch reinfocement learning automatisiert verschieden starke Computergegner aus einer Simulation zu extrahieren.
%Gehören schon immer zum menschlichen miteinander... barrieren abbauen


\section{Problemstellung}
%Was steht zur Verfügung? Was ist das Problem genau(mathematisch)? Random vs best move every turn~
%genau das spiel beschreiben und was das problem dann ist und wie das gelöst werden kann
Ein Computergegner der für das Spiel Vier Gewinnt erschaffen wird kann auf verschiedenen Algorithmen basieren er könnte zum Beispiel in jedem Zug einfach zufällig eines der sieben möglichen Einwurflöcher bedienen oder er könnte durch einen Minimax Algorithmus zu jedem Zeitpunkt den bestmöglichen Zug errechnen. 
Das Problem hieran ist, dass jeder einzelne Algorithmus programmiert werden muss und somit ein enormer Arbeitsaufwand entsteht.
Ein Lösungsansatz für diese Problem sind selbstlernende Algorithmen die mit Hilfe von reinforement learning und neuronalen Netzen selbstständig ähnlich wie ein Mensch erlernen wie das Spiel zu gewinnen ist. Hierraus können dann automatisch Computergegner generiert werden. 

\section{Umsetzung}
%Wie gehe ich vor?
%Minimax algorithmus als gegner erstellen und an dem dann trainieren
%minimax da für dieses spiel möglich

Es wird eine digitale Version des Spiels Vier Gewinnt genutzt und eine Schnittstelle erschafft die es ermöglicht auszuwählen ob ein Mensch oder der Computer die Rolle der Spieler übernimmt. Dies ist wichtig da ein selbstlernender Algorithmus viele Spiele durchlaufen muss um einen Lernfortschritt aufzuzeigen. Somit kann in der Simulation dann trainiert werden ohne jedes einzelne Spiel gegen einen Menschen spielen zu müssen. Spielt der Computer soll zwischen verschieden effektiven Varianten von einem Minimax Algorithmus ausgewählt werden können. Somit kann der reinforcement learning Agent schnell viele Spiele gegen einen guten Gegner spielen und so effektiv lernen. 
Der Agent wird mit verschiedenen Voraussetzungen lernen um vergleichen zu können was besser und schlechter beim lernen hilft.


%Wie wird das am Ende evaluiert? 
%was macht hier sinn? betreuer nochmal fragen
\section{Evaluation}
Je nachdem wie gut der selbstlernende Agent nach dem Lernen gegen unterschiedlich effektive Varianten des Minimax Algorithmus ist zeigt dann wie gut das reinforcement learning funktioniert hat.
Hieran wird dann verglichen wie gut verschiedene Techniken des reinforcement learnings sind um den bestmöglichen Agenten zu erschaffen.
Endgültig soll sich dann zeigen, dass durch das reinforcement learning Gegner auf automatisierte Weise geschaffen werden können die einen fein granularen Schwierigkeitsanstieg bieten können. \colorbox{red!30}{TODO}

\section{Related work}
Ich habe zwei Arbeiten gefunden die auch mit Hilfe von reinforcement learning versuchen, das Spiel Vier Gewinnt zu meistern. Einmal die Arbeit von Lukas Stephan, welcher das Problem mit einer Java Implementierung angegagen ist und den Lernalgorithmus mit einer Mustererkennung ausgestattet hat. Zweitens die Arbeit von Markus Thill, welcher N-Tupel-Systeme nutzt, um seinen Agenten zu Trainieren. Anders als diese Arbeiten ist mein Ansatz nicht das Erkennen von Mustern, sondern das Lernen durch das Verhalten des Gegners.
%related work???
\colorbox{red!30}{TODO Cite}










\chapter{Grundlagen}%

\label{cha:Schluss}

In dieser Arbeit werden einige Technologien genutzt um die Problemstellung zu lösen.
Die wichtigsten dieser Technologien werden hier erklärt um das weitere Verständniss des Lesers zu gewährleisten. 


\section{Vier Gewinnt}
Vier Gewinnt ist ein Spiel für 2 Spieler in dem abwechselnd Spielsteine in ein vertikales Spielfeld eingeworfen werden. Jeder Spieler kann sich in seinem Zug zwischen einem von sieben Einwurflöchern entscheiden. Wird ein Spielstein eingeworfen so fällt dieser auf die niedrigste der sechs Positionen die noch nicht durch andere Spielsteine gefüllt ist. Sind schon sechs Spielsteine in ein bestimmtest Einwurfloch gesteckt wurden so darf hier kein weiterer Spielstein eingeworfen werden. Ein Spieler gewinnt das Spiel wenn er es geschafft  hat, dass sich vier seiner Spielsteine in einer Reihe befinden. Eine Reihe kann horizontal, vertikal oder diagonal gebildet werden. Werden alle 42 Positionen mit Spielsteinen befüllt ohne eine Reihe von vier Steinen des selben Spielers zu bilden geht die Partie unentschieden aus.

\section{Reinforcement Learning}
Reinforcement Learning (Bestärkendes Lernen) ist eine Maschine-Learning Methode bei der ein Agent mit Hilfe eines Environments, Policy, State, Actions und eines Rewards lernt eine ihm gegebene Aufgabe zu lösen.
Die Aufgabe ist in diesem Fall das Spiel Vier Gewinnt zu meistern.

\subsection{Agent}
Bei dem Agenten handelt es sich um den Acteur der die ihm gegebene Aufgabe meistern soll. Er sieht das Environment und wählt mittels der Policy die ihm am besten erscheinende Action aus.
Hier ist der Agent einer der beiden Spieler von Vier Gewinnt.

\subsection{Environment}
Das Environment beschreibt das gesamte Problem dass der Agent versucht zu lösen.
In Vier Gewinnt ist dies das Spielfeld und der Gegner. 

\subsection{Policy}
Mit der Policy wird das Verhalten beschrieben nach dem der Agent entscheidet welche der möglichen Actions die aktuel beste ist um den Reward zu maximieren.\\
TODO %welche wird genommen?

\subsection{State}
Der State beschreibt den actuellen Zustand des Environments.
Für Vier Gewinnt beschreibt der State welche der 42 Felder mit welchen Steinen befüllt sind. 

\subsection{Actions}
Actions sind die möglichen Aktionen zwischen denen sich der Agent je nach State entscheiden muss.
Die Actions in Vier Gewinnt beschreiben die sieben Löcher in die der Agent Steine werfen kann.

\subsection{Reward}

\subsection{psychologischer Hintergrund}
Reinforcement Learning funktioniert nach dem Prinzip von Trial-and-Error.\\
TODO %ausführen

\subsection{Q-Funktion}
TODO %ausführen


\section{Software}
Um die Problemstellung zu bearbeiten wurde eine Implementierung des Spiels in Python genutzt und aus dieser mit openai gym ein Environment geschaffen. An diesem Environment trainiert dann ein Agenten der durch Keras-RL erschaffen wurde.

\subsection{Python}
TODO
\subsection{Tensorflow}
TODO
\subsection{Keras-RL}
TODO
\chapter{Stand der Technik}
\label{cha:Technik}

In den letzten Jahrzehnten wurde für viele Spiele, die bis dahin als zu komplex galten, Programme geschaffen, die diese spielen können. Das Spielniveau der hierdurch erschaffenen Spieler ist dabei gleich gut oder besser als das von den besten menschlichen Spielern.
Dies liegt zu einem großen Teil an den Fortschritten in der Hardware, welche es ermöglichen, große künstliche neronale Netzwerke zu erstellen und zu trainieren. Einige der bekanntesten Erfolge sind zum Beispiel Deep Blue für Schach und Alpha Go für das Spiel Go.\cite{Sutton2018}\\
Weitere Erfolge aus den letzten Jahren sind Agenten, die Computerspiele meistern, wie zum Beispiel MarI/O \cite{mario}, welcher Super Mario spielt oder AlphaStar \cite{alphastar}, welcher Starcraft II spielen kann.
Die Veröffentlichung von OpenAI Gym\cite{gym}, welche unter anderem eine Menge an Atari-Spielen zur Verfügung stellt, eröffnete das Thema Reinforcement-Learning für Spiele für die gesamte Welt. 
Wie in dem Buch von Sutton und Barto \cite{Sutton2018} gezeigt wurde, lassen sich Markow-Entscheidungsprobleme mit Hilfe von reinforcement learning lösen.   
Dass das Lernen durch das Imitieren des Verhalten eines Profis verbessert werden kann, wurde unter anderm von Oliver Amantier gezeigt. \cite{NIPS2016_6391,ARMANTIER2004221,price1999implicit}


\section{Arbeiten mit ähnlichem Inhalt}
Ich habe zwei Arbeiten gefunden, die auch mit Hilfe von Reinforcement-Learning versuchen, das Spiel Vier Gewinnt zu meistern. Einmal die Arbeit von Lukas Stephan \cite{Stephan2018}, welcher das Problem mit einer Java Implementierung angegagen ist und den Lernalgorithmus mit einer Mustererkennung ausgestattet hat. Zweitens die Arbeit von Markus Thill, Patrick Koch und Wolfgang Konen \cite{Thill2012}, welche N-Tupel-Systeme nutzten, um einen Agenten zu trainieren. Anders als diese Arbeiten ist mein Ansatz nicht das Erkennen von Mustern, sondern das Lernen durch das Verhalten des Gegners.










\chapter{Konzept}%
       

\label{cha:Konzept}

Wie soll alles gemacht werden


\section{Plan}



\chapter{Implementierung}%

\label{cha:Implementierung}

Vorgehensweise bei der Implementierung


\section{Code Grundlage}
Verwendeter Code 


\chapter{Evaluierung}%

\label{cha:Eval}

Was kann evaluiert werden? \colorbox{red!30}{Muss jetzt anhand der veränderung vom Agneten gemacht werden}\\
\colorbox{red!30}{jede veränderung vergleichen}


\section{Auswertung}
\colorbox{red!30}{Ergebnisse in Text übersetzen-> Graph, Durchschnitt,...}

\colorbox{red!30}{Schlüsse die aus den Daten gezogen werden können -> Signifikantstest?}





\chapter{Fazit}%

\label{cha:Fazit}

Was wurde festgestellt

\section{Besonderheiten}



%%%----------------------------------------------------------
\appendix                                            % Anhang 
%%%----------------------------------------------------------


%%%----------------------------------------------------------
\MakeBibliography                        % Quellenverzeichnis
%%%----------------------------------------------------------

%%% Messbox zur Druckkontrolle ------------------------------
%\chapter*{Messbox zur Druckkontrolle}



\begin{center}
{\Large --- Druckgröße kontrollieren! ---}

\bigskip

\calibrationbox{100}{50} % Angabe der Breite/Hoehe in mm

\bigskip

{\Large --- Diese Seite nach dem Druck entfernen! ---}

\end{center}



%%%----------------------------------------------------------
\end{document}
%%%----------------------------------------------------------